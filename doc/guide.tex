\documentclass{article}



\usepackage{hyperref}
\hypersetup{
	colorlinks,
	citecolor=black,
	filecolor=magenta,
	linkcolor=blue,
	urlcolor=cyan
}



\title{PARSEC 1.4 User's Guide}
\author{
{Prof. J. R. Chelikowsky's Research Group} \\
{http://real-space.org/}
}
\date{\today}

\makeindex

\begin{document}

\maketitle
\tableofcontents






\section{What it is}

PARSEC is a computer code that solves the Kohn-Sham equations by 
expressing electron wave-functions directly in real space, without use 
of explicit basis set. The name stands for "Pseudopotential Algorithm 
for Real-Space Electronic Calculations". It handles one-dimensional and 
three-dimensional periodic boundary conditions and confined-system 
boundary conditions. In the latter, the charge density and 
wave-functions are required to vanish outside a spherical shell 
enclosing the electronic system. A finite-difference approach is used 
for the calculation of spatial derivatives. Norm-conserving 
pseudopotentials are used to describe the interaction between valence 
electrons and ionic cores (core electrons+nuclei). Owing to the sparsity 
of the Hamiltonian matrix, the Kohn-Sham equations are solved by direct 
or iterative diagonalization, with the use of extremely efficient 
sparse-matrix eigensolvers. Besides standard DFT calculations, parsec is 
able to perform structural relaxation, simulated annealing, Langevin 
molecular dynamics, and polarizability calculations (confined-system 
boundary conditions only). It can also be used to provide input for 
TDLDA (time-dependent density functional theory, in adiabatic 
local-density approximation), GW approximation, and solution of the 
Bethe-Salpeter equation for optical excitations.

The development of parsec started in the early 1990's with the work of
 J.R. Chelikowsky, N. Troullier,
 and Y. Saad ( Phys. Rev. Lett {\bf 72}, 1240 ; Phys. Rev. B {\bf 50},
11355). Later, many people have contributed with additional features and
performance improvement:

\begin{itemize}
\item A. Stathopolous : mathematical subroutines;
\item H. Kim : core correction, multipole expansion;
\item I. Vasiliev : Polarizability, asymptotically-corrected functionals;
\item M. Jain and L. Kronik : GGA, generalized Broyden mixer,
  ``cleaning'' of the source code, efficiency improvements, and added functions;
\item K. Wu : improvements in the iterative eigensolver {\em diagla} 
      and the Hartree solver by conjugate-gradient;
\item R. Burdick : major rewrite and improvement of multipole expansion;
\item M. Alemany and M. Jain : periodic boundary conditions, parallel
implementation;
\item M. Alemany and E. Lorin : ARPACK solver;
\item A. Makmal: dynamical memory allocation;
\item M. Tiago : parallelization, symmetry operations, wire (1-D)
  boundary conditions, this user's guide;
\item Y. Zhou : Chebyshev-Davidson eigensolver; 
thick-restart Lanczos eigensolver; Chebyshev subspace
filtering method which serves to avoid diagonalizations;
\item A. Natan : non-orthogonal boundary conditions; slab (2-D)
\item A. Benjamini: slab (2-D)
\item D. Nave : spin-orbit potential;
\item H.-W. Kwak : LDA+U;
\item S. Beckman : k-point sampling.
\end{itemize}







\section{Getting the code}

Public versions of the code can be downloaded from
{Prof. Chelikowsky's group page} 
{http://real-space.org/}. There is also a development
version, with implementations that are being tested for stability and
performance. If you are interested in contributing to development,
contact the developers.

PARSEC is distributed under the General Public License (GPL). In a
nutshell, these are the conditions implied in using it:

\begin{itemize}
\item It should be used for non-commercial purposes only.
\item The authors give no warranty regarding its reliability or
  usefulness to your particular application.
\item The authors offer no technical support.
\end{itemize}



\subsection{Compiling PARSEC}

At the beginning of the {\tt Makefile} file, edit the include line with
the correct machine-dependent compilation file. Existing compilation files
are inside subdirectory {\tt config}. Some of the existing files are:

\begin{itemize}
\item {\tt make.ibmsp3, make.ibmsp3\_mpi} : sp.msi.umn.edu, with/without MPI support.
\item {\tt make.intel} : plain i686 machine running linux.
\end{itemize}

This code has been ported and tested on a variety of different 
platforms: IBM SP3 and SP4 (AIX compiler), SGI (intel and f90 
compilers), Intel/Linux (g95, intel compilers).  Once the {\tt 
config/make.*} file has been edited, build the executable by running 
make.  Upon completing a successful build, the file {\tt parsec.ser} (no 
MPI support) or {\tt parsec.mpi} (with MPI support) will be found in the 
src directory.

PARSEC uses several external libraries. For the basic serial build the
only necessary external library is LAPACK/BLAS. The best option is to
pick machine optimized versions of these libraries such as ESSL (IBM)
or MKL (Intel). If these are unavailable you can download LAPACK from
\href{http://www.netlib.org/lapack/}{netlib}. If you want to use
periodic boundary conditions then
\href{http://www.fftw.org/}{FFTW} is needed. Turn this feature on
by passing either -DUSEFFTW3 or -DUSEFFTW2 to the preprocessor
depending on if you are using the version 2 or 3 of the library. If
you want to use the parallel implementation you need
\href{http://www-unix.mcs.anl.gov/mpi/mpich/}{MPICH2}. To make the
parallel version pass -DMPI to the preprocessor. If you want to use
the ARPACK diagonalization method, you need to build the
\href{http://www.caam.rice.edu/software/ARPACK/}{ARPACK library}. Pass the flag
-DUSEARPACK to the preprocessor.

To port PARSEC to new machines you will need to modify the existing 
config/make.* file for the machine you want to use. In addition you may 
need to modify the machine specific files {\tt myflush.f90p}, {\tt 
timing.f90p}, {\tt cfftw.f90p}. If you look in these you will notice 
that there have been a variety of ports made and machine specific 
modifications have been specified by preprocessor calls (e.g. the 
preprocessor flag {\tt -DIBM} specifies that the IBM ``mclock'' function 
be used for timing calls).  In the case that the Portland Group compiler 
is used, the preprocessor flag {\tt -DPGI} must be passed.

In addition to these preprocessor flags there are several other
options that can be passed to the preprocessor. These are generally
for development or other highly specialized purposes. Their operation
can be learned by reading the source code. 



\subsection{Porting to new platforms}

Besides creating a new {\tt make.*} file for the machine you want to use,
these are the files you may have to modify in
order to port it to other platforms: myflush.F, timing.F, cfftw.F

PARSEC uses the following external libraries:

\begin{itemize}
\item \href{http://www.caam.rice.edu/software/ARPACK/}{ARPACK}, diagonalization of sparse matrices. Optional. The interface with arpack is turned on with compilation flag {\tt  USEARPACK}.

\item \href{http://www.fftw.org}{FFTW}, versions 2.x or 3.x,
used for 3-dimensional FFT of real functions. Optional. The interface
with FFTW is turned on with compilation flag {\tt  USEFFTW2} or {\tt USEFFTW3}. Notice that, without FFTW, PARSEC cannot handle periodic systems.

\item \href{http://www.netlib.org/lapack/}{LAPACK/BLAS},
linear algebra. Alternatively, support for libraries 
ESSL (IBM) and \href{http://www.intel.com/cd/software/products/asmo-na/eng/index.htm}{MKL} (Intel Math Kernel Library) is also available.

\item MPI : message passing interface, for parallel machines.

\end{itemize}











\section{Running PARSEC}



\subsection{Basic input}

\begin{itemize}
\item {\tt parsec.in} : input file with user-provided parameters.

\label{InputPseudopotentials}
\item Pseudopotential files. For the moment, only  norm-conserving
  pseudopotentials are implemented. Currently, PARSEC supports four
  different formats (see {\tt Pseudopotential\_Format}
  {PseudopotentialFormat}):

\begin{itemize}
\item {\tt Martins\_old }: original format, included for
  back-compatibility with older versions of PARSEC. Pseudopotential files have
  name {\tt *\_POTR.DAT}. In this format, files do not have atomic parameters
  such as cut-off radii included.

\item {\tt Martins\_new }: format of files created with
  Prof. J.L. Martins' pseudopotential generator,
  \href{http://bohr.inesc-mn.pt/~jlm/pseudo.html}{version 5.69p}. The distributors of
  Parsec have a version of Prof. Martins' code with minor but
  important modifications, so that files in the proper format are
  created. Pseudopotential files have name {\tt  *\_POTRE.DAT}.
  Files in this format are compatible with the
  \href{http://www.uam.es/siesta/}{``Siesta''} DFT
  program, but the opposite is not true: pseudo-wavefunctions are
  appended to {\tt *\_POTRE.DAT} files, but the same does not
  necessarily happen to formatted pseudopotential files used in
  Siesta. WARNING: different from {\tt Martins\_old} format, a single
  radial grid is used for pseudopotentials, pseudowavefunctions and
  densities, and the grid density has a default value which is often too
  low. In some cases, this can be a problem, especially for the core
  density. There are at least two solutions for this problem: either
  modify the default value of grid density (for example, set {\tt bb=160}
  in file input.f), or specify your own grid density in input file
  {\tt atom.dat}, or the equivalent in your pseudopotential code. A
  good sanity check is to compare the results provided by parsec with
  reliable results obtained by using other DFT codes.

\item {\tt Martins\_Wang }: format of files created with Martins' code
  as modified by Lin-Wang Wang. This format is also supported by the
  DFT codes \href{http://hpcrd.lbl.gov/~linwang/PEtot/PEtot.html}{PEtot} and
  \href{http://www.nersc.gov/projects/paratec/}{PARATEC}. Pseudopotential files have
  name {\tt *\_POTRW.DAT}.

\item {\tt FHIPP }: format of files created by the
  \href{http://www.fhi-berlin.mpg.de/th/fhi98md/fhi98PP/}{FHIPP code}. Pseudopotential
  files have name {\tt *\_FHIPP.DAT}. This format allows calculations
  within the virtual crystal approximation, but they do not currently
  allow the inclusion of spin-orbit corrections.
\end{itemize}

\end{itemize}

We do not offer any advice to the user about the choice of
pseudopotential, although we do distribute the newest version of
J.L. Martins pseudopotential generator with PARSEC. For generating
pseudopotentials we suggest the user read the webpages:

\begin{itemize}

\item \href{http://www.fhi-berlin.mpg.de/th/fhi98md/fhi98PP/}{http://www.fhi-berlin.mpg.de/th/fhi98md/fhi98PP/},
\item \href{http://opium.sourceforge.net/}{http://opium.sourceforge.net/},
\item \href{http://bohr.inesc-mn.pt/~jlm/pseudo.html}{http://bohr.inesc-mn.pt/~jlm/pseudo.html} and
\item \href{http://hpcrd.lbl.gov/~linwang/PEtot/PEtot.html}{http://hpcrd.lbl.gov/~linwang/PEtot/PEtot.html}.
\end{itemize}



\subsection{The parsec.in file}

The user-provided input uses the electronic structure data format 
(ESDF), implemented as a module written by Chris J. Pickard. Comments 
can be added on any line, and are preceded by any of the symbols ``{\tt 
\# ; !}''. Keywords and blockwords are case insensitive. Characters 
``{\tt . : =}'' and multiple spaces are ignored. Most parameters have 
default values. All parameters with true/false value have default value 
false. Parameter values can be written anywhere in the file, with the 
exception that information pertinent to the same chemical element should 
be given in consecutive lines (sometimes, data from different elements 
can be mixed, but this should be avoided for the sake of clarity). 
Keywords are always followed by their corresponding value on the same 
line. Input values with intrinsic physical unit can be given in 
different units, with name following the value. Default unit is used if 
none is specified. For example, the line

\begin{verbatim}
grid_spacing 0.5 ang
\end{verbatim}
%
sets a grid spacing of 0.5 angstroms. Blocks have the syntax:

\begin{verbatim}
begin Blockword
.
.
.
end Blockword
\end{verbatim}

WARNINGS: 

\begin{itemize}

\item PARSEC does not watch out for misprints in {\tt parsec.in}. Lines
with unrecognized keywords or blockwords are usually ignored.

\item The EDSL library sometimes screws up when identifying
certain features with certain atom types (more on this in
chapter \ref{KnownProblems}).
\end{itemize}



\subsection{Basic output}

\begin{itemize}
\item {\tt parsec.out} : The unified output file. This file contains
  most of the relevant messages about the calculation as well as the
  results.

\item {\tt out.*} : The standard output from each processor. It
  contains some warnings and error messages. Information about the
  diagonalization as well as other processor specific information is
  written here. These files are useful for debugging.

\item {\tt parsec.dat} : The eigenvalues, density function, effective
  potential, wave-functions and other output data, in binary
  (unformatted) form. This file can sometimes be very lengthy. 
\end{itemize}



\subsection{Interrupting calculation \label{StopSCF}}

In situations when the user wants to interrupt a calculation, he/she
can do it by creating a file with name {\tt stop\_scf} (case sensitive). This file can
be empty, but it must be created in the working directory, the same
one which contains all the input/output files. The sequence of
SCF iterations will be aborted as early as possible and output written
to files {\tt parsec.out} and {\tt parsec.dat}. The last file can be
used then as input to a subsequent calculation. This feature is useful
if the user wants to stop the calculation ``smoothly'' and be able to
restart it at a later time.








\section{Input options}



\subsection{The real-space grid}

The following parameters define the region of space where the
Kohn-Sham equations will be solved, and the appropriate boundary
conditions. In parallel environment, grid points are distributed among
processors. Symmetry operations are used to reduce the sampled region
to an irreducible wedge.

\subsubsection{\tt Boundary\_Conditions
\index{Boundary\_Conditions@{\tt Boundary\_Conditions}}
\label{BoundaryConditions}}
({\it word}),
default value = {\tt cluster }

Defines the type of boundary conditions to be imposed. Available
options are:

    \begin{itemize}
    \item{\tt cluster}. Specifies completely confined system, with boundary
    conditions such that the electron wavefunctions are zero outside a spherical
    domain. The discretized grid is laid out following the Cartesian
    directions. If necessary (and if
    consistent with other input options), a constant vector is added to all
    atomic coordinates so that the geometric center of the systems coincides with
    the origin of the Cartesian system.
    
    \item {\tt bulk}. Specifies periodic boundary conditions along all
    3 Cartesian (or generalized) directions. If necessary, the input grid
    spacing may be increased so as to
    make it commensurable with the given size of periodic cell. Also, atomic
    coordinates are used as given, with no initial repositioning around the origin
    of the real-space grid. If atoms are found to be outside the cell, they are
    replaced by their periodic image inside the cell.
    
    \item {\tt wire}. The
    system is assumed to be periodic along the Cartesian {\bf x} direction
    and confined within a cylinder centered on the {\bf x} axis. The
    periodicity length is defined in block {\tt Cell\_Shape}
    {CellShape}. The cylinder radius is defined by {\tt
    Boundary\_Sphere\_Radius} {BoundarySphereRadius}. This type of
    boundary conditions can be used together with k-point sampling.
    
    \item {\tt slab}. The system is assumed to be periodic along the
    Cartesian {\bf x} and {\bf y} directions. The cell structure is defined
    in the block {\tt Cell\_Shape} {CellShape}. The slab width is 
    twice the value that is defined in  {\tt Boundary\_Sphere\_Radius} 
    {BoundarySphereRadius}. This boundary conditions can be used together with
    k-point sampling. 
    \end{itemize}



\subsubsection{\tt Cell\_Shape 
\index{Cell\_Shape@{\tt Cell\_Shape}}
\label{CellShape}}
({\it block}),
default value = {\tt no default }

Specification of unit lattice vectors in Cartesian coordinates. Notice
that, in contrast to earlier versions, this version requires the full
vectors. With bulk boundary conditions (periodicity along the 3
Cartesian directions), this block must have three lines, for the
coordinates of the 3 unit lattice vectors. With wire boundary
conditions (periodicity along the {\bf x} direction only, this block
must have one line with a single number: the periodicity length along
{\bf x}. With slab boundary conditions (periodicity along the {\bf x} and 
{\bf y}), this block must have 2 lines with 2d vectors that span the x-y
plane, if 3d vevtors are given, the 3rd number is ignored.  A global scale of unit lattice vectors can be done with input
flag ({\tt Lattice\_Vector\_Scale}{LatticeVectorScale}).
Relevant only in periodic systems (bulk, slab, or wire). 

\subsubsection{\tt Lattice\_Vector\_Scale 
\index{Lattice\_Vector\_Scale@{\tt Lattice\_Vector\_Scale}}
\label{LatticeVectorScale}}
({\it real}),
default value = {\tt 1 },
default unit = {\it bohr} (Bohr radii)

Constant which multiplies the length of all three unit lattice
vectors. This is useful for volumetric expansions or contractions in
periodic systems. Relevant only in periodic systems.

\subsubsection{\tt Super\_Cell 
\index{Super\_Cell@{\tt Super\_Cell}}
\label{SuperCell}}
({\it block}),
default value = {\tt 1 1 1 }

Specification of super-cell with respect to unit lattice vectors of a
periodic system. If declared to be ($n,m,l$), then the actual periodic
cell will be a $n \times m \times l$ expansion of the cell specified in
input. Atom coordinates are expanded accordingly. This is useful in
super-cell calculations where the super-cell is obtained by
replicating a minimal cell several times. Not used in non-periodic
systems. In wire (1-D) systems, only the first number is used. In slab 
(2-D) systems, only the first 2 numbers are used.

\subsubsection{\tt Super\_Cell\_Vac 
\index{Super\_Cell\_Vac@{\tt Super\_Cell\_Vac}}
\label{SuperCellVac}}
({\it block}),
default value = {\tt 0 0 0 }

Specification of vacuum in a supercell calculation. If declared to be
($n,m,l$), the super-cell is expanded by adding $n$ empty cell along the
first periodic direction, $m$ empty cells along the second periodic
direction and $l$ empty cells along the third periodic direction. If
applicable, this expansion is done after the expansion defined by
{\tt Super\_Cell}{SuperCell}. Not used in non-periodic
systems. In wire (1-D) systems, only the first number is used. In slab 
(2-D) systems, only the first 2 numbers are used.

\subsubsection{\tt Boundary\_Sphere\_Radius 
\index{Boundary\_Sphere\_Radius@{\tt Boundary\_Sphere\_Radius}}
\label{BoundarySphereRadius}}
({\it real}),
default value = {\tt no default },
default unit = {\it bohr} (Bohr radii)

Radius of enclosing spherical shell. Boundary conditions in
non-periodic systems are such that all wave-functions vanish beyond a spherical
shell centered in the origin and with given radius. The Hartree potential is
also set to vanish on the shell (internally, this is done by solving Poisson's
equation with a compensating charge outside the shell). Calculation stops if any
atom is found outside the shell. The amount of extra space that should be given
between the shell and any inner atom is strongly system-dependent and should be
carefully tested for convergence. With wire boundary conditions
(system periodic along the {\bf x} axis and confined along
perpendicular directions), this option defines the radius of the
confining cylinder. With slab boundary conditions this number is half the slab
width in the {\bf z} axis.

\subsubsection{\tt Grid\_Spacing 
\index{Grid\_Spacing@{\tt Grid\_Spacing}}
\label{GridSpacing}}
({\it real}),
default value = {\tt no default },
default unit = {\it bohr} (Bohr radii)

Distance between neighbor points in the real-space grid.
This is a critical parameter in the calculation and it should be carefully
tested for convergence. The grid is assumed to be regular. In periodic systems 
(3-D and 2-D) the grid can be non-orthogonal according to the lattice vectors
definition.

\subsubsection{\tt Expansion\_Order 
\index{Expansion\_Order@{\tt Expansion\_Order}}
\label{ExpansionOrder}}
({\it integer}),
default value = {\tt 12 }

Order used in the finite-difference expansion of gradients and Laplacians. It
corresponds to twice the number of neighbor points used on each side and for
each direction. It must be an even number.

\subsubsection{\tt Ignore\_Symmetry 
\index{Ignore\_Symmetry@{\tt Ignore\_Symmetry}}
\label{IgnoreSymmetry}}
({\it boolean}),
default value = {\tt false}

Cancels all symmetry operations. if this flag is true, symmetry is not going 
to be used in the calculation. 

\subsubsection{\tt Double\_Grid\_Order 
\index{Double\_Grid\_Order@{\tt Double\_Grid\_Order}}
\label{DoubleGridOrder}}
({\it integer}),
default value = {\tt 1 }

Order for the double-grid used in the calculation of pseudopotentials
in real space. In order to account for short-range variations in
pseudopotentials (which can break translational symmetry in periodic
systems), a double grid is defined following the method of Ono and
Hirose ({\it Phys. Rev. Lett.} {\bf 82}, 5016 (1999) ). A double-grid
order $N$ implies that $N-1$ additional grid points will be added
between two neighbor points in the original grid, and Kohn-Sham
eigenvectors interpolated between them. The default value is
equivalent to original setting (no double grid).

\subsubsection{\tt Cluster\_Domain\_Shape
\index{Cluster\_Domain\_Shape@{\tt Cluster\_Domain\_Shape}}
\label{ClusterDomainShape}}
({\it word}),
default value = {\tt sphere}

This option is relevant for the 0-D case only. It defines the shape of
the boundary conditions of the grid. Implemented options are:

\begin{itemize}
\item {\tt sphere}: Spherical boundary conditions are used. The domain is
defined by all the points that fulfill $(x^2+y^2+z^2)<R^2$ where R is defined
by the {\tt Boundary\_Sphere\_Radius} {BoundarySphereRadius}

\item {\tt ellipsoid}: Ellipsoid boundary conditions are used. The 
Ellipsoid paramaters are defined by the {\tt Domain\_Shape\_Parameters} 
{DomainShapeParameters} block, this block MUST be defined.

\item {\tt cylinder}: Cylindrical boundary conditions are used. The cylinder
parameters are defined by the {\tt Domain\_Shape\_Parameters}
{DomainShapeParameters} block, this block MUST be defined.

\item {\tt box}: Box boundary conditions defines a rectangular cuboid box. The
box parameters are defined by the {\tt Domain\_Shape\_Parameters}
{DomainShapeParameters} block, this block MUST be defined.
\end{itemize}

\subsubsection{\tt Domain\_Shape\_Parameters
\index{Domain\_Shape\_Parameters@{\tt Domain\_Shape\_Parameters}}
\label{DomainShapeParameters}}
({\it block}), 
default value = {\tt no default}

This block defines the paramers for the differnet shapes of the grid. A
single line with 3 numbers is supplied. The meaning of the numbers depend
on the domain definition:

\begin{itemize}
\item{ellipsoid}: The numbers are simply the a,b,c of the ellipsoid. 

\item{cylinder}: The numbers are R, L, d. R is the cylinder radius, L is
the cylinder length and d is the orientation: d=1 for x, d=2 for y and d=3
for Z. 

\item{box}: The numbers are simply a,b,c - the box dimensions in x,y,z 
respectively.
\end{itemize} 
 
\subsubsection{\tt Full\_Hartree
\index{Full\_Hartree@{\tt Full\_Hartree}}
\label{FullHartree}}
( {\it boolean}),
default value = {\tt false }

This flag is set for the case of an isolated shape that is extended 
in one dimension relative to the others (e.g. a cylinder where $L >> R$). 
In such shapes the usual multipole approximation is not converging well 
and so an exact calculation of the Hartree potential is performed for 
the boundary conditions of the Poisson equation. Note that such calculation 
is much slower than the usual multipole approximation.
   
\subsubsection{\tt Boundary\_Mask 
\index{Boundary\_Mask@{\tt Boundary\_Mask}}
\label{BoundaryMask}}
({\it boolean}),
default value = {\tt false }

Option available in confined systems only. Allows for flexibility in
the definition of the real-space grid. If enabled, the grid will be
constructed according to function {\tt My\_Mask} built in the
source. See source for more information. WARNING: function {\tt
  My\_Mask} must be tailored to the specific system.















\subsection{Diagonalization parameters}

Many DFT codes solve the Kohn-Sham equations by minimizing
the total energy functional. Parsec uses a different approach: it
performs iterative diagonalization of the Kohn-Sham
equations. Expressed in real space, the Hamiltonian is a sparse
matrix, and efficient algorithms can be used to obtain the low-energy
eigenstates with minimal memory usage and parallel communication.

\subsubsection{\tt Eigensolver 
\index{Eigensolver@{\tt Eigensolver}}
\label{Eigensolver}}
({\it word}),
default value = {\tt chebdav }

Implemented eigensolvers are:

\begin{itemize}
\item {\tt arpack}:
Implicitly restarted Lanczos eigensolver as implemented in ARPACK.
Reference: \href{http://www.caam.rice.edu/software/ARPACK/}
{http://www.caam.rice.edu/software/ARPACK/}
\item {\tt diagla}: Block preconditioned Lanczos.
Reference: A. Stathopoulos, Y. Saad, and K. Wu, SIAM J. on Scientific
Computing {\bf 19} 229 (1998).
\item {\tt trlanc}: Locally modified 
thick-restart Lanczos package developed by K. Wu and H. Simon.
Reference: {{http://crd.lbl.gov/~kewu/trlan.html}
{http://crd.lbl.gov/~kewu/trlan.html}}
\item {\tt chebdav}: Chebyshev-Davidson eigensolver algorithm. References:
Y. Zhou, Y. Saad, M.L. Tiago, and J.R. Chelikowsky,
J. Comput. Phys. {\bf 219}, 172 (2006); Phys. Rev. E {\bf 74}, 066704 (2006).
\item {\tt chebff}: Chebyshev-filtered subspace iteration eigensolver. References:
Y. Zhou, J. R. Chelikowsky, Y. Saad,
J. Comput. Phys. {\bf 274}, 770--782 (2014).
\end{itemize}

The last two eigensolvers use significantly less memory, and
{\tt diagla} occasionally fails to converge in periodic systems.

\subsubsection{\tt Diag\_Tolerance 
\index{Diag\_Tolerance@{\tt Diag\_Tolerance}}
\label{DiagTolerance}}
({\it real}),
default value = {\tt 1.d-4 }

Tolerance in the residual norm of eigen-pairs.

\subsubsection{\tt Maximum\_Matvec 
\index{Maximum\_Matvec@{\tt Maximum\_Matvec}}
\label{MaximumMatvec}}
({\it integer}),
default value = {\tt max(30000,N*neig) }

Maximum number of matrix-vector operations to be allowed on each
diagonalization. This is useful if diagonalization fails to
converge. The ideal value depends on the actual number of rows in the 
Hamiltonian ({\tt N}) and the number of desired eigenvalues ({\tt neig}).
Calculations with hundreds of atoms or more may require this
parameter to be increased from its default value.

\subsubsection{\tt Subspace\_Buffer\_Size 
\index{Subspace\_Buffer\_Size@{\tt Subspace\_Buffer\_Size}}
\label{SubspaceBufferSize}}
({\it integer}),
default value = {\tt 5 }

Number of additional eigenvalues to be computed for each representation. Must
be some positive value, so that the full set of eigenvalues can be
reconstructed from the sum of representations without missing levels.

\subsubsection{\tt Chebyshev\_Filtering 
\index{Chebyshev\_Filtering@{\tt Chebyshev\_Filtering}}
\label{ChebyshevFiltering}}
({\it boolean}),
default value = {\tt false }

After the first few diagonalizations, this option replaces the
standard diagonalization with a polynomial filtering of the
approximate eigen-subspace. 
This technique is usually much more efficient than any of the
eigensolvers above. It is enabled automatically if the eigensolver
{\tt chebdav} is used. Reference for the subspace filtering method is:

Y. Zhou, Y. Saad, M.L. Tiago, and J.R. Chelikowsky,
{\it Self-consistent-field calculation using Chebyshev
polynomial filtered subspace iteration}, J. Comput. Phys. {\bf 219}, 172 (2006)

\subsubsection{\tt Chebdav\_Degree 
\index{Chebdav\_Degree@{\tt Chebdav\_Degree}}
\label{ChebdavDegree}}
({\it integer}),
default value = {\tt 20 }

Degree of Chebyshev polynomial used in the Chebyshev-Davidson
eigensolver. The degree is recommended to be within 15 to 30,
usually a degree within 17 to 25 should work well.
%It should not be less than 15, and better results are
%obtained with polynomial 20 or more.

\subsubsection{\tt Chebyshev\_Degree 
\index{Chebyshev\_Degree@{\tt Chebyshev\_Degree}}
\label{ChebyshevDegree}}
({\it integer}),
default value = {\tt 15 }

Average degree of Chebyshev polynomial used in the subspace filtering
method, this degree generally is smaller than the degree used
in the Chebyshev-Davidson method. 
Note that the subspace filtering method serves to avoid 
doing diagonalizations, while the Chebyshev-Davidson method is
used for diagonalization. 

This  polynomial degree is recommended to be within 7 to 20.  
If slow convergence is observed for some hard problems, 
one may use a higher degree, e.g., 21 to 25.

\subsubsection{\tt Chebyshev\_Degree\_Delta
\index{Chebyshev\_Degree\_Delta@{\tt Chebyshev\_Degree\_Delta}}
\label{ChebyshevDegreeDelta}}
({\it integer}),
default value = {\tt 1 } (if {\tt Chebyshev\_Degree} $<$ 10), {\tt 3} otherwise

During subspace filtering, the energy spectrum is divided in two windows of
equal width: the low-energy window is filtered with polynomial of degree
$p-\delta$ and the high-energy window is filtered with polynomial of degree
$p+\delta$, where $p$ is the average degree ({\tt Chebyshev\_Degree}) and
$\delta$ is the degree variation ({\tt Chebyshev\_Degree\_Delta}). Large
values of the degree variation usually give shorter runtime, but they may
make the filtering process unstable if the ratio $\delta/p$ is much larger
than around 0.30.


\subsection{Self-consistency and mixing parameters}

\subsubsection{\tt Max\_Iter 
\index{Max\_Iter@{\tt Max\_Iter}}
\label{MaxIter}}
({\it integer}),
default value = {\tt 50 }

Maximum number of iterations in the self-consistent loop.

\subsubsection{\tt Convergence\_Criterion 
\index{Convergence\_Criterion@{\tt Convergence\_Criterion}}
\label{ConvergenceCriterion}}
({\it real}),
default value = {\tt 2.d-4 },
default unit = {\it Ry} (rydbergs)

The density function is considered converged when the self-consistent
residual error (SRE) falls below this value. The SRE is an integral of the
square of the difference between the last two self-consistent potentials,
weighted by electron density and taken squared root. This parameter should
not be chosen less than the typical diagonalization accuracy.

\subsubsection{\tt Convergence\_Criterion\_Approach
\index{Convergence\_Criterion\_Approach@{\tt Convergence\_Criterion\_Approach}}
\label{ConvergenceCriterionApproach}}
({\it real}),
default value = {\tt 100 * Convergence\_Criterion },
default unit = {\it Ry} (rydbergs)

If the SRE becomes less than this value and if the polynomial order during
subspace filtering is more than 10, the polynomial order will be reduced
by one unity after each SCF cycle until the order becomes equal to 10.
Beyond that point, the order is kept fixed. This is a performance parameter,
and it exploits the fact that subspace filtering with low order is as efficient
as filtering with high order if the eigenvectors are close enough to the
converged solution.

\subsubsection{\tt Mixing\_Method 
\index{Mixing\_Method@{\tt Mixing\_Method}}
\label{MixingMethod}}
({\it word}),
default value = {\tt Anderson }

Defines the scheme used in the mixing of old and new potentials. 
Implemented schemes are {\tt Broyden}, {\tt Anderson}, {\tt msecant1}, 
{\tt msecant2}, {\tt msecant3}. The first two methods were proved to be 
equivalent by V. Eyert, 1996. In practice, there is no significant 
difference between them. Broyden mixing uses disk for input/output of 
internal data. The last three methods are multi-secant methods. They 
were developed and coded by Haw-ren Fang and Yousef Saad, at the 
University of Minnesota. {\tt Msecant1} is a class of multi-secant 
methods with Type-I update to minimize the change of Jacobian. {\tt 
Msecant2} is a class of multi-secant methods with Type-II update to 
minimize the change of inverse Jacobian. {\tt Msecant3} is a hybrid 
method, that minimizes the change of Jacobian or inverse Jacobian 
depending on the secant errors. This last method was inspired by the 
work of J.M. Mart{\'i}nez.

\subsubsection{\tt Mixing\_Param 
\index{Mising\_Param@{\tt Mixing\_Param}}
\label{MixingParam}}
({\it real}),
default value = {\tt 0.30 }

Choice for the diagonal Jacobian used in the mixing of potentials. Typical
range is between {\tt 0.1} and {\tt 1.0}. Small values reduce the amount
of mixing and result in slower but more controlled convergence. For improved
performance, the user may try reducing this parameter in cases where
the energy landscape is too corrugated (e.g.: magnetic systems, or systems
with many degrees of freedom).

\subsubsection{\tt Memory\_Param 
\index{Memory\_Param@{\tt Memory\_Param}}
\label{MemoryParam}}
({\it integer}),
default value = {\tt 4 }

Number of previous SCF steps used in mixing. Ideal values range from 4 to 6.
Larger values may actually worsen convergence because it includes steps far
from the real solution.

\subsubsection{\tt Restart\_Mixing 
\index{Restart\_Mixing@{\tt Restart\_Mixing}}
\label{RestartMixing}}
({\it integer}),
default value = {\tt 20 }

Maximum number of iterations in the self-consistent loop before mixing
is restarted. This should improve SCF convergence in cases when mixing
becomes ``stuck'' and the SRE stops decreasing.

\subsubsection{\tt Mixing\_Group\_Size
\index{Mising\_Group\_Size@{\tt Mixing\_Group\_Size}}
\label{MixingGroupSize}}
({\it integer}),
default value = {\tt 1 }

Sets the size of groups of secant equations. To set infinite
size of groups, use {\tt Mixing\_Group\_Size: 0}.

\subsubsection{\tt Mixing\_EN\_Like
\index{Mising\_EN\_Like@{\tt Mixing\_EN\_Like}}
\label{MixingENLike}}
({\it integer}),
default value = {\tt 0 }

Indicates using the Broyden-like update of the approximate
(inverse) Jacobian. For EN-like update, set {\tt Mixing\_EN\_Like: 1}.
Possible values are {\tt 0} and {\tt 1}.

\subsubsection{\tt Mixing\_Preferred\_Type
\index{Mising\_Preferred\_Type@{\tt Mixing\_Preferred\_Type}}
\label{MixingPreferredType}}
({\it integer}),
default value = {\tt 1 }

Relevant only for hybrid multi-secant methods ({\tt msecant3}). Possible
values are {\tt 0} and {\tt 1}. The preferred type of update should be
specified when the secant error is undefined (i.e., the first group of update).

\subsubsection{\tt Mixing\_Restart\_Factor
\index{Mising\_Restart\_Factor@{\tt Mixing\_Restart\_Factor}}
\label{MixingRestartFactor}}
({\it real}),
default value = {\tt 0.10 }

This parameter defines the restarting factor (ratio between the the norm of
old function versus new function). If $||f_{new}||$ is too large relative
to $||f_{old}||$, then the linear model or its approximation is not
reliable, so restart should be performed. More precisely,
if $||f_{old}|| < \mbox{restart\_factor} * ||f_{new}||$ then restart should be
performed. In particular, if restart\_factor is 0, then no restart is done.

\subsubsection{\tt Mixing\_Memory\_Expand\_Factor
\index{Mising\_Memory\_Expand\_Factor@{\tt Mixing\_Memory\_Expand\_Factor}}
\label{MixingMemoryExpandFactor}}
({\it real}),
default value = {\tt 2.0 }

This line means that whenever the memory for mixing is insufficient, it will
expand by a factor of 2.0, or by the value defined at input.


The last five parameters are only effective for multi-secant
methods. Anderson and Broyden methods are special cases of multi-secant
methods. One
can reduce to these methods by choosing carefully the multi-secant parameters.
For example, Broyden's first method is obtained by setting:

\begin{verbatim}
Mixing_Group_Size: 1
Mixing_EN_Like: 0
Mixing_Method: msecant1
\end{verbatim}
%
whereas Broyden's second method is obtained by replacing {\tt msecant1} above
with {\tt msecant2}. Anderson mixing is obtained by setting:

\begin{verbatim}
Mixing_Method: msecant2
Mixing_Group_Size: 0
Mixing_EN_Like: 0
\end{verbatim}

The very first nonlinear Eirola-Nevanlinna (EN)-like algorithm proposed by Yang
can be obtained by setting:

\begin{verbatim}
Mixing_Method: msecant1
Mixing_Group_Size: 1
Mixing_EN_Like: 1
\end{verbatim}








\subsection{Global atom and pseudopotential parameters}

\subsubsection{\tt Atom\_Types\_Num 
\index{Atom\_Types\_Num@{\tt Atom\_Types\_Num}}
\label{AtomTypesNum}}
({\it integer}),
default value = {\tt no default }

Number of different chemical elements present in the system.

\subsubsection{\tt Coordinate\_Unit 
\index{Coordinate\_Unit@{\tt Coordinate\_Unit}}
\label{CoordinateUnit}}
({\it word}),
default value = {\tt Cartesian\_Bohr }

By default, positions of atoms are given in cartesian coordinates and
units of Bohr radius. Other implemented options are:

\begin{itemize}
\item {\tt Cartesian\_Ang} atomic coordinates given in angstroms, cartesian.

\item {\tt Lattice\_Vectors} atomic coordinates given in units of lattice
  vectors. Available only with periodic boundary conditions.
\end{itemize}

\subsubsection{\tt Old\_Interpolation\_Format 
\index{Old\_Interpolation\_Format@{\tt Old\_Interpolation\_Format}}
\label{OldInterpolationFormat}}
({\it boolean}),
default value = {\tt false }

This is a compatibility flag. Normally, the local component of pseudopotentials
{\it V} at some arbitrary point {\it r} is evaluated by linear interpolation of the
function {\it r*V}. Due to the Coulomb-like behavior of {\it V} far away from
the atom, this interpolation is exact beyond the cut-off radius. This
flag reverts to the old method: linear interpolation of {\it V} itself.
It should be used for debugging purposes only.








\subsection{Parameters specific to each atom type}

These parameters should be given in sequence in {\tt parsec.in}, and
they refer to a particular chemical element in the system.

\subsubsection{\tt Atom\_Type 
\index{Atom\_Type@{\tt Atom\_Type}}
\label{AtomType}}
({\it word}),
default value = {\tt no default }

Chemical symbol of the current element. THIS IS CASE SENSITIVE!

\subsubsection{\tt Pseudopotential\_Format 
\index{Pseudopotential\_Format@{\tt Pseudopotential\_Format}}
\label{PseudopotentialFormat}}
({\it word}),
default value = {\tt Martins\_new }

Specification for the format of input pseudopotentials. Available
choices are: {\tt Martins\_old}, {\tt Martins\_new}, {\tt Martins\_Wang}
and {\tt FHIPP}. See also \linebreak { input of pseudopotential 
files}{InputPseudopotentials}.

\subsubsection{\tt Potential\_Num 
\index{Potential\_Num@{\tt Potential\_Num}}
\label{PotentialNum}}
({\it integer}),
default value = {\tt no default }

Number of available angular components. Relevant for pseudopotential
formats {\tt Martins\_old} and {\tt FHIPP}.

\subsubsection{\tt Core\_Cutoff\_Radius 
\index{Core\_Cutoff\_Radius@{\tt Core\_Cutoff\_Radius}}
\label{CoreCutoffRadius}}
({\it real}),
default value = {\tt no default },
default unit = {\it bohr} (Bohr radii)

Maximum cut-off radius used in the generation of pseudopotential.
Relevant for pseudopotential formats {\tt Martins\_old}, {\tt
  Martins\_Wang} and {\tt FHIPP}.

\subsubsection{\tt Electron\_Per\_Orbital 
\index{Electron\_Per\_Orbital@{\tt Electron\_Per\_Orbital}}
\label{ElectronPerOrbital}}
({\it block}),
default value = {\tt no default }

Number of electrons per atomic orbital used in the generation of
pseudopotential. Relevant for pseudopotential formats {\tt
  Martins\_old} and {\tt FHIPP}.

\subsubsection{\tt Local\_Component 
\index{Local\_Component@{\tt Local\_Component}}
\label{LocalComponent}}
({\it integer}),
default value = {\tt no default }

Choice of angular component to be considered local in the Kleinman-Bylander
transformation. No ghost-state check is done in parsec. Therefore, check if
the chosen component does not have ghost states. Relevant for
pseudopotential formats {\tt Martins\_old}, {\tt Martin\_new} and {\tt FHIPP}.

\subsubsection{\tt Nonlinear\_Core\_Correction 
\index{Nonlinear\_Core\_Correction@{\tt Nonlinear\_Core\_Correction}}
\label{NonlinearCoreCorrection}}
({\it boolean}),
default value = {\tt false }

This flag should be enabled if the pseudopotentials for this element
have non-linear core correction. Relevant for pseudopotential format
{\tt FHIPP} only.

\subsubsection{\tt Move\_Flag 
\index{Move\_Flag@{\tt Move\_Flag}}
\label{MoveFlag}}
({\it integer}),
default value = {\tt 0 }

Type of movement to be applied to atoms. Relevant only if structural relaxation
is performed. Value {\tt 0} makes all atoms movable in all directions. Value
{\it n}
positive makes the first {\it n} atoms movable. A negative value makes selected atoms
movable (the selection is made inside block {\tt Atom\_Coord}
{AtomCoord}).

\subsubsection{\tt Atom\_Coord 
\index{Atom\_Coord@{\tt Atom\_Coord}}
\label{AtomCoord}}
({\it block}),
default value = {\tt no default }

Coordinates of atoms belonging to the current chemical species. See
{\tt Coordinate\_Unit} {CoordinateUnit}. If structural relaxation is done with selected atoms
({\tt Move\_Flag} {MoveFlag} $<$ 0), each triplet of coordinates
must be followed by an integer: 0 if this atom is fixed, 1 if it is movable.

\subsubsection{\tt Alpha\_Filter 
\index{Alpha\_Filter@{\tt Alpha\_Filter}}
\label{AlphaFilter}}
({\it real}),
default value = {\tt 0.0 }

Cut-off parameter in Fourier filtering of pseudopotentials, in reciprocal
space. If value is non-positive, pseudopotentials are not
filtered. Fourier filtering is implemented according to Briggs {\it et
  al.} Phys. Rev. B {\bf 54}, 14362 (1996), but no second filtering in
real space has been implemented so far.

\subsubsection{\tt Beta1\_Filter 
\index{Beta1\_Filter@{\tt Beta1\_Filter}}
\label{Beta1Filter}}
({\it real}),
default value = {\tt 0.0 }

Cut-off parameter in Fourier filtering of pseudopotentials, in reciprocal
space. Referenced only if {\tt alpha\_filter} is positive.

\subsubsection{\tt Core\_Filter 
\index{Core\_Filter@{\tt Core\_Filter}}
\label{CoreFilter}}
({\it  real}),
default value = equal to {\tt Alpha\_Filter}

Cut-off parameter in Fourier filtering of non-linear core density. If
non-positive, then core density is not filtered.

\subsubsection{\tt Initial\_Spin\_Polarization
\index{Initial\_Spin\_Polarization@{\tt Initial\_Spin\_Polarization}}
\label{InitialSpinPolarization}}
({\it real}),
default value = {\tt 0.1 },

Value of spin polarization to be used in the initial guess of charge
density. An initial spin polarization of 0.1 means that the charge
ratio $( n^\uparrow - n^\downarrow )/( n^\uparrow + n^\downarrow )$ is
0.1. Warning: if you specify this parameter for one chemical element,
then you also must specify it for all the other ones (use zero for
non-spin polarized atoms).

\subsubsection{\tt LDAplusU\_U 
\index{LDAplusU\_U@{\tt LDAplusU\_U}}
\label{LDAplusUU}}
({\it real}),
default value = {\tt 0.0 },
default unit = {\it Ry} (rydbergs)

Value of the U parameter (on-site Coulomb interaction) for this
chemical element. A positive value for this quantity or for the J
parameter below will add a on-site Coulomb interaction in the DFT
Hamiltonian, following the ``LDA+U'' approach, see: Anisimov {\it et
  al.}, Phys. Rev. B {\bf 44}, 943 (1991).

\subsubsection{\tt LDAplusU\_J 
\index{LDAplusU\_J@{\tt LDAplusU\_J}}
\label{LDAplusUJ}}
({\it real}),
default value = {\tt 0.0 },
default unit = {\it Ry} (rydbergs)

Value of the J parameter (exchange part of on-site Coulomb
interaction) for this chemical element.

\subsubsection{\tt SO\_psp 
\index{SO\_psp@{\tt SO\_psp}}
\label{SOpsp}}
({\it boolean}),
default value = {\tt false }

Enables the inclusion of spin-orbit components in the
pseudopotentials for this chemical element. Notice that {\tt FHIPP}
format does not currently have support for spin-orbit
components. Spin-orbit components for pseudopotential format
{\tt Martins\_old} require one additional file, with name
{\tt *\_SO.DAT}. Notice that the inclusion of spin-orbit components
require spin-polarized calculation, flag
{\tt Spin\_Polarization}{SpinPolarization}.

\subsubsection{\tt SCF\_SO 
\index{SCF\_SO@{\tt SCF\_SO}}
\label{SCFSO}}
({\it boolean}),
default value = {\tt false }

Enables the inclusion of spin-orbit components as
perturbation. Spin-orbit components are included in a SCF calculation
only after the solution is obtained without spin-orbit
components. This can lead to easier convergence.

\subsubsection{\tt Cubic\_Spline 
\index{Cubic\_Spline@{\tt Cubic\_Spline}}
\label{CubicSpline}}
({\it boolean}),
default value = {\tt false }

Enables cubic spline interpolation in the pseudopotentials. This is
useful if the radial grid where pseudopotentials are computed is
coarse, especially in the vicinity of the cut-off radius.












\subsection{Electronic parameters}

\subsubsection{\tt Restart\_Run 
\index{Restart\_Run@{\tt Restart\_Run}}
\label{RestartRun}}
({\it boolean}),
default value = {\tt false}

By default, the initial charge density in a parsec run is built from
the superposition of atomic charge densities. If you want instead to
start from a previous run, choose value true.
For confined systems, having {\tt restart\_run = true} prevents the initial
centering of atoms. If this flag is used but an adequate file {\tt
  parsec.dat} is not found, the calculation is interrupted.

\subsubsection{\tt States\_Num 
\index{States\_Num@{\tt States\_Num}}
\label{StatesNum}}
({\it integer}),
default value = {\tt no default }

Number of energy eigenvalues to be computed. In spin-polarized calculations,
this is the number of eigenvalues per spin orientation.

\subsubsection{\tt Net\_Charges 
\index{Net\_Charges@{\tt Net\_Charges}}
\label{NetCharges}}
({\it real}),
default value = {\tt 0.0 },
default unit = {\it e} (electron charge)

Net electric charge in the system, in units of electron charge. It can also
be used in periodic systems. In that case, a background charge is used
and the divergent component of the Hartree potential is ignored.

\subsubsection{\tt Fermi\_Temp 
\index{Fermi\_Temp@{\tt Fermi\_Temp}}
\label{FermiTemp}}
({\it real}),
default value = {\tt 80.0 },
default unit = {\it K} (kelvins)

Fermi temperature used to smear out the Fermi energy. If a
negative value is given, then the occupancy of Kohn-Sham orbitals is read from
file {\tt occup.in} and kept fixed throughout the
calculation. 

\subsubsection{\tt Correlation\_Type 
\index{Correlation\_Type@{\tt Correlation\_Type}}
\label{CorrelationType}}
({\it word}),
default value = {\tt CA } (without spin polarization), {\tt PL } (with
spin polarization)

Choice of exchange-Correlation functional to be used. Available
choices and their variants are:

\begin{itemize}
\item {\tt CA, PZ }: Ceperley-Alder (LDA), with spin-polarized
  option. D.M. Ceperley and B.J. Alder, Phys. Rev. Lett. {\bf 45},
  566 (1980);  J.P. Perdew and A. Zunger, Phys. Rev. B {\bf 23}, 
5048 (1981).

\item {\tt XA}: Slater's x-alpha (LDA). J.C. Slater, Phys. Rev. {\bf 81} 385 (1951).

\item {\tt WI}: Wigner's interpolation (LDA). E. Wigner, Phys. Rev. {\bf 46},
  1002 (1934).

\item {\tt HL}: Hedin-Lundqvist (LDA). L. Hedin and B.I. Lundqvist,
  J. Phys. C {\bf 4}, 2064 (1971).

\item {\tt LB}: Ceperley-Alder with Leeuwen-Baerends correction
  (asymptotically corrected). R. van Leeuwen and E. J. Baerends,
  Phys. Rev. A {\bf 49}, 2421 (1994).

\item {\tt CS}: Ceperley-Alder with Casida-Salahub correction
  (asymptotically corrected). M.E. Casida and D.R. Salahub,
  J. Chem. Phys. {\bf 113}, 8918 (2000).

\item {\tt PL, PWLDA}: Perdew-Wang (LDA), with spin-polarized
  option. J.P. Perdew and Y. Wang, Phys. Rev. B {\bf 45}, 13244 (1992).

\item {\tt PW, PW91, PWGGA}: Perdew-Wang (GGA), with spin-polarized
  option. J. Perdew, {\it Electronic Structure of Solids}
  (ed. P. Ziesche and H. Eschrig, Akademie Verlag, Berlin, 1991).

\item {\tt PB, PBE}: Perdew-Burke-Ernzerhof (GGA), with spin-polarized
  option. J.P. Perdew, K. Burke, and M. Ernzerhof, Phys. Rev. Lett.
  {\bf 77}, 3865 (1996).
\end{itemize}

\subsubsection{\tt Ion\_Energy\_Diff 
\index{Ion\_Energy\_Diff@{\tt Ion\_Energy\_Diff}}
\label{IonEnergyDiff}}
({\it real}),
default value = {\tt 0.0},
default unit = {\it Ry} (rydbergs)

Energy difference used in the construction of the Casida-Salahub functional.
Relevant only if that functional is used.

\subsubsection{\tt Spin\_Polarization 
\index{Spin\_Polarization@{\tt Spin\_Polarization:}}
\label{SpinPolarization}}
({\it boolean}),
default value = {\tt false }

Enables spin-polarized densities and functional.

\subsubsection{\tt Symmetrize\_Charge\_Density 
\index{Symmetrize\_Charge\_Density@{\tt Symmetrize\_Charge\_Density:}}
\label{SymmetrizeChargeDensity}}
({\it boolean}),
default value = {\tt false }

Forces charge density to be resymmetrized after each SCF step. In some
cases (for example, if the Fermi temperature is too small), the charge
density may lose symmetry after some steps due to round-off error
accumulation. This option is valid only in confined (totally
non-periodic) systems. For periodic systems, the charge density is
always symmetrized.

\subsubsection{\tt Add\_Point\_Charges 
\index{Add\_Point\_Charges@{\tt Add\_Point\_Charges:}}
\label{AddPointCharges}}
({\it boolean}),
default value = {\tt false }

Enables embedding of the system in electrostatic point charges. This
feature can be used when the actual system is surrounded by additional
ions which do not take part in electronic processes and can be taken
into account as classical point charges, without a full
quantum-mechanical treatment. Also, the system must be confined:
{\tt Boundary\_Conditions cluster}{BoundaryConditions}.

\subsubsection{\tt Point\_Typ\_Num 
\index{Point\_Typ\_Num@{\tt Point\_Typ\_Num}}
\label{PointTypNum}}
({\it integer}),
default value = {\tt 0}

Number of classical point charge types. Referenced only if point
charge embedding is enabled.

\subsubsection{\tt Pt\_Chg 
\index{Pt\_Chg@{\tt Pt\_Chg}}
\label{PtChg}}
({\it real}),
default value = {\tt 0.0},
default unit = {\it e} (electron charge)

Charge of each point charge type.

\subsubsection{\tt Point\_Coord 
\index{Point\_Coord@{\tt Point\_Coord}}
\label{PointCoord}}
({\it block}),
default value = {\tt no default}

Coordinates of each point charge. They can also be rescaled with a
coordinate unit. See {\tt Coordinate\_Unit} {CoordinateUnit}.

\subsubsection{\tt External\_Mag\_Field 
\index{External\_Mag\_Field@{\tt External\_Mag\_Field}}
\label{ExternalMagField}}
({\it real}),
default value = {\tt 0.0},
default unit = {\it Ry/mu\_Bohr} (rydbergs per Bohr magneton)

Strength of the external magnetic field applied on the system, along
the {\bf z} direction. This magnetic field acts only on the spin magnetic
dipole, and not on orbital magnetic dipole (orbital angular momentum
is very small for electrons).









\subsection{Wave functions and k-points}

The parameters here affect the wave functions and the k-point
sampling.

\subsubsection{\tt Complex\_Wfn 
\index{Complex\_Wfn@{\tt Complex\_Wfn}}
\label{ComplexWfn}}
({\it boolean}),
default value = {\tt true}

By default use complex value for the wavefunctions.  This is necessary
for sampling of k-space and for other instances.  For the case that 
only the $\Gamma$ point is sampled the calculation will run much
faster and uses fewer resources if this parameter is turned to false.

\subsubsection{\tt Kpoint\_Method 
\index{Kpoint\_Method@{\tt Kpoint\_Method}}
\label{KpointMethod}}
({\it string}),
default value = {\tt none }

The method to sample k-space.  There are two options
\begin{itemize}
\item{\tt none}: No additional k-point is used besides the $\Gamma$
  (0,0,0) point. Also, wave-functions are computed as real functions
  (null imaginary part).

\item{\tt mp}:  Sample by the Monkhorst-Pack algorithm (see
  H.J. Monkhorst and J.D. Pack, Phys. Rev. B {\bf 13}, 5188 (1976))
This will require including a Monkhorst\_Pack\_Grid in a block.

\item{\tt manual}:  Manual sampling of k-space.  This will require
  including a kpoint\_list in a block.

\end{itemize}

\subsubsection{\tt Monkhorst\_Pack\_Grid 
\index{Monkhorst\_Pack\_Grid@{\tt Monkhorst\_Pack\_Grid}}
\label{MonkhorstPackGrid}}
({\it block}),
default value = {\tt none }

Grid to sample kpoints over via the Monkhorst-Pack algorithm. Relevant
only if  {\tt Kpoint\_Method}{KpointMethod} = {\tt mp}.

\subsubsection{\tt Monkhorst\_Pack\_Shift 
\index{Monkhorst\_Pack\_shift@{\tt Monkhorst\_Pack\_Shift}}
\label{MonkhorstPackShift}}
({\it block}),
default value = {\tt none }

Shift of Monkhorst-Pack grid. Kpoints in this grid have coordinates
given by: $k_i = ( i + s )/n$ in units of reciprocal lattice vectors,
where n is a grid order (see {\tt Monkhorst\_Pack\_Grid}
{MonkhorstPackGrid}), s is a shift and i is an integer with value
between 0 and $n-1$. Relevant
only if  {\tt Kpoint\_Method}{KpointMethod} = {\tt mp}.

\subsubsection{\tt Kpoint\_List 
\index{Kpoint\_List@{\tt Kpoint\_List}}
\label{KpointList}}
({\it block}),
default value = {\tt none }

List of k-points to sample and the sampling weight of each k-point. Relevant only if {\tt
  Kpoint\_Method}{KpointMethod} = {\tt manual}.

\subsubsection{\tt Kpoint\_Unit 
\index{Kpoint\_Unit@{\tt Kpoint\_Unit}}
\label{KpointUnit}}
({\it string}),
default value = {\tt Cartesian\_Inverse\_Bohr }

Defines the type of coordinates used to input k-points. Available
options are {\tt Cartesian\_Inverse\_Bohr } (Cartesian coordinates,
units of inverse Bohr radius) and {\tt Reciprocal\_Lattice\_Vectors }
(units of reciprocal lattice vectors).











\subsection{Band structure and Density of States (DOS) calculation}

Special commands are given for band structure calculation and density 
of states (DOS) calculation. Both calculations can be carried out at the
end of an SCF run - so no need for a special run. 

\subsubsection{\tt bandstruc
\index{bandstruc@{\tt bandstruc}}
\label{bandstruc}}
({\it block}),
default value = {\tt no default }

bandstruc is a block containing the path in k-space to be calculated 
for the band structure. The block is composed of lines. Each line defines 
a segment in the band path and contains 7 numbers and a label, the first 
number is a serial number. the next 6 numbers contain the starting k-point 
of the segment and final k-point of the segment, given in {\it relative} 
coordinates. the last part of each line is a label, describing the segment. 

An example is shown below for the 2d case of graphene:

\begin{verbatim}
begin bandstruc
1 0.0 0.0 0.0 0.5 0.0 0.0 gamma-M
2 0.5 0.0 0.0 0.333333333 -0.333333333 0.0 M-K
3 0.333333333 -0.333333333 0.0 0.0 0.0 0.0 K-gamma
end bandstruc
\end{verbatim}

In the case of 2d - the 3rd number of each k-point should be $0.0$. The 
band structure calculation is done after the SCF calculation has ended. 
At the end of calculation a text file that is called "bands.dat" is created.
The file is arranged a line of number. Each line is arranged as follows:

\begin{verbatim}

spin, line_num, kpoint_num, kpt_x, kpt_y, kpt_z, eigs(kpt,spin,1:n)-efermi

\end{verbatim}

spin - is simply 1 or 2, line\_number is the path segment number, kpoint\_num
is the k-point number, kpt\_x, kpt\_y and kpt\_z are the k-point components,
note that they are given in {\it cartesian} coordinates and not in relative. 
eigs are simply the eigenvalues, they are given relative to the fermi level. 

\subsubsection{\tt bandstruc\_points
\index{bandstruc\_points@{\tt bandstruc\_points}}
\label{BandstrucPoints}}
({\it integer}),
default value = {\tt 20 }

This parameter defines the number of points in the {\it shortest} segment
in the band path that is given. The calculation is done so that the points
density along the path is kept constant.  

\subsubsection{\tt create\_dos
\index{create\_dos@{\tt create\_dos}}
\label{CreateDos}}
({\it boolean}),
default value = {\tt false}

Density of States (DOS) is calculated when this flag is true. DOS calculation
can be performed only for periodic (3d) or partially periodic (1d and 2d) 
systems.

\subsubsection{\tt dos\_pnum
\index{dos\_pnum@{\tt dos\_pnum}}
\label{DosPnum}}
({\it integer}),
default value = {\tt 1000}

Number of points in the energy axis for the DOS calculation. 


  





\subsection{Non-self consistent calculations and BerkeleyGW interface}

Special commands are given for non-self consistent calculations and interface
to the {\tt BerkeleyGW} code {http://www.berkeleygw.org}. Unlike
band structure and density of states (DOS) calculation, the non-self consistent
calculations has to be carried out after the end of an SCF run. The code
reads {\tt parsec.dat} file for charge density and potential information.
This file is NOT overwritten.  It should be kept in mind that one has to 
specify the parameters for the scf calculation as well. The code uses 
these when reading {\tt parsec.dat} to ensure that it is the intended
file. The non-self consistent code is independent of the BerkeleyGW interface
and can be used separately as well. 
 
\subsubsection{\tt nscf\_kpoints
\index{nscf\_kpoints@{\tt nscf\_kpoints}}
\label{nscfkpoints}}
({\it block}),
default value = {\tt no default }

nscf\_kpoints is a block containing the points in k-space and their weights
to be calculated in the non-self consistent calclation. The weights are not 
used unless one is outputing information for BerkeleyGW calculation. The format
for each line in this block is the same as {\tt Kpoint\_list}.

\subsubsection{\tt Nscf\_Kpoint\_Unit
\index{Nscf\_Kpoint\_Unit@{\tt Nscf\_Kpoint\_Unit}}
\label{NscfKpointUnit}}
({\it string}),
default value = {\tt Cartesian\_Inverse\_Bohr }

Defines the type of coordinates used to input {\tt nscf\_kpoints}. Available
options are {\tt Cartesian\_Inverse\_Bohr } (Cartesian coordinates,
units of inverse Bohr radius) and {\tt Reciprocal\_Lattice\_Vectors }
(units of reciprocal lattice vectors).

\subsubsection{\tt nscf\_states\_num
\index{nscf\_states\_num@{\tt nscf\_states\_num}}
\label{nscfstatesnum}}
({\it integer}),
default value = {\tt none }

This parameter defines the number of states to be calculated in the 
non-self consistent calculation.

\subsubsection{\tt nscf\_fermi\_level
\index{nscf\_fermi\_level@{\tt nscf\_fermi\_level}}
\label{nscffermilevel}}
({\it real}),
default value = {\tt none}

Because no self-consistent calculation is done in the same run, the code 
does not have any idea about the Fermi Level. So a Fermi level has to be 
provided from the previous scf calculation. Default units are Ry.

\subsubsection{\tt output\_gw
\index{output\_gw@{\tt output\_gw}}
\label{outputgw}}
({\it boolean}),
default value = {\tt false }

This flag determines whether output for {\tt BerkeleyGW} is written. In
order to use this flag, the code must be compiled with {-DBGW} flag. It 
also has to be linked properly to {\tt BerkeleyGW} code libraries which
contain subroutines for input/output in {\tt BerkeleyGW} format. When true,
this flag will write {\tt WFN}, {\tt VXC}, and {\tt RHO} files. These 
files contain the wavefunction information (in g-space), exchange-correlation
potential (in g-space) and charge density (in g-space) respectively. These
can be used as input to the {\tt BerkeleyGW} code.

\subsubsection{\tt nscf\_kgrid
\index{nscf\_kgrid@{\tt nscf\_kgrid}}
\label{nscfkgrid}}
({\it block}),
default value = {\tt none }

Grid used to sample kpoints via the Monkhorst-Pack algorithm for non-self 
consistent calculation. This information is just passed to {\tt BerkeleyGW}. 
The calculation is performed on the {\tt nscf\_kpoints}{nscfkpoints}. 
Relevant only if  {\tt output\_gw}{outputgw} = {\tt true}.

\subsubsection{\tt nscf\_kgrid\_shift
\index{nscf\_kgrid\_shift@{\tt nscf\_kgrid\_shift}}
\label{nscfkgridshift}}
({\it block}),
default value = {\tt none }

Shift of Monkhorst-Pack grid for non-self consistent calculations. 
Kpoints in this grid have coordinates given by: $k_i = ( i + s )/n$ 
in units of reciprocal lattice vectors, where n is a grid order 
(see {\tt nscf\_kgrid}{nscfkgrid}), s is a shift and i is an 
integer with value between 0 and $n-1$. This information is just passed 
to {\tt BerkeleyGW}. The calculation is performed on the 
{\tt nscf\_kpoints}{nscfkpoints}. Relevant only if 
{\tt output\_gw}{outputgw} = {\tt true}.







\subsection{Structural relaxation}

\subsubsection{\tt Minimization 
\index{Minimization@{\tt Minimization}}
\label{Minimization}}
({\it word}),
default value = {\tt none }

Choice of method for structural relaxation. Available options are:
{\tt none} (no
relaxation), {\tt simple} (steepest-descent algorithm), {\tt BFGS}
(Broyden-Fletcher-Goldfarb-Shanno, the best), and {\tt manual} (manual movement). Steepest-descent should be
used as last resort. With manual movement, the sequence of atomic coordinates
must be provided by the user in file {\tt manual.dat} and it is used
``as is''. The last choice is useful for debugging. Symmetry operations
are ignored if manual movement is performed

\subsubsection{\tt Movement\_Num 
\index{Movement\_Num@{\tt Movement\_Num}}
\label{MovementNum}}
({\it integer}),
default value = {\tt 500 }

Maximum number of iterations during structural relaxation.

\subsubsection{\tt Force\_Min 
\index{Force\_Min@{\tt Force\_Min}}
\label{ForceMin}}
({\it real}),
default value = {\tt 1.0d-2 },
default unit = {\it Ry/bohr} (rydbergs/Bohr radii)

Maximum magnitude of the force on any atom (that was allowed to move) in the relaxed structure.
This should not be set less than about 0.01 {\it Ry/a.u.}, which is the
usual accuracy in the calculation of forces.

\subsubsection{\tt Max\_Step 
\index{Max\_Step@{\tt Max\_Step}}
\label{MaxStep}}
({\it real}),
default value = {\tt 0.05} (steepest descent), {\tt -1} (BFGS),
default unit = {\it bohr} (Bohr radii)

If steepest-descent minimization is performed, this parameter sets the
maximum displacement of atoms between two steps in relaxation.
With BFGS, this defines the maximum displacement of atoms along
each Cartesian component from their original position (if negative, no
constraint in the movement of atoms is imposed). Ignored in all other
circumstances.

\subsubsection{\tt Min\_Step 
\index{Min\_Step@{\tt Min\_Step}}
\label{MinStep}}
({\it real}),
default value = {\tt 1.d-4 },
default unit = {\it bohr} (Bohr radii)

Minimum displacement of atoms between two steps in steepest-descent 
relaxation. Relevant only if steepest descent is performed.

\subsubsection{\tt BFGS\_Number\_Corr 
\index{BFGS\_Number\_Corr@{\tt BFGS\_Number\_Corr}}
\label{BFGSNumberCorr}}
({\it integer}),
default value = {\tt 7 }

Number of variable metric corrections used to define the limited-memory
matrix. Relevant only if BFGS relaxation is performed.

\subsubsection{\tt Relax\_Restart 
\index{Relax\_Restart@{\tt Relax\_Restart}}
\label{RelaxRestart}}
({\it boolean}),
default value = {\tt false }

Flag for the restart of a relaxation run. If true, the user must provide 
information about previous run in file {\tt relax\_restart.dat}. No need 
to provide input file {\tt parsec.dat}. If this option is used, the 
atomic coordinates in {\tt parsec.in} are replaced with the ones in {\tt 
relax\_restart.dat}. Appropriate warnings are written to output. If the 
restart file is not found, this flag is ignored and no information from 
previous relaxations is used.









\subsection{Molecular dynamics}

\subsubsection{\tt Molecular\_Dynamics 
\index{Molecular\_Dynamics@{\tt Molecular\_Dynamics}}
\label{MolecularDynamics}}
({\it boolean}),
default value = {\tt false }

Enables molecular dynamics. This feature can be used both for
simulated annealing and for direct simulation of hot systems. Both
canonical and micro-canonical ensembles are implemented. Symmetry
operations are ignored if molecular dynamics is being performed.

\subsubsection{\tt Cooling\_Method 
\index{Cooling\_Method@{\tt Cooling\_Method}}
\label{CoolingMethod}}
({\it word}),
default value = {\tt stair }

Type of cooling (i.e.: change of ensemble temperature) used during molecular
dynamics. Possible choices are: {\tt stair} (stair-case), {\tt log}
(logarithmic cooling), and {\tt linear} (linear cooling). As the
molecular dynamics simulation proceeds, the
ensemble temperature changes according to the cooling method and to the
choices of initial and final temperatures. At each step, the actual temperature
is calculated from equipartition theorem and the kinetic energy (difference
between the total energy and potential energy). In the first steps, the
actual and ensemble temperatures may be very different, particularly if there
are few atoms, when thermal fluctuations are enormous.

\subsubsection{\tt Tinit 
\index{Tinit@{\tt Tinit}}
\label{Tinit}}
({\it real}),
default value = {\tt 500.0 },
default unit = {\it K} (kelvins)

Initial ensemble temperature in the system.

\subsubsection{\tt T\_final 
\index{T\_final@{\tt T\_final}}
\label{Tfinal}}
({\it real}),
default value = {\tt 500.0 },
default unit = {\it K} (kelvins)

Final ensemble temperature.

\subsubsection{\tt T\_Step 
\index{T\_Step@{\tt T\_Step}}
\label{TStep}}
({\it real}),
default value = {\tt 150.0 },
default unit = {\it K} (kelvins)

Step temperature used in the cooling. Used only with stair-case
cooling method.

\subsubsection{\tt Time\_Step 
\index{Time\_Step@{\tt Time\_Step}}
\label{TimeStep}}
({\it real}),
default value = {\tt 150.0 },
default unit = {\it $\hbar$/hartrees} ($0.0242 fs = 2.42 \times 10^{-17}$ seconds)

Time interval between two steps in molecular dynamics.
The dynamical equations of motion are numerically integrated using a modified
Verlet algorithm. 

\subsubsection{\tt Friction\_Coefficient 
\index{Friction\_Coefficient@{\tt Friction\_Coefficient}}
\label{FrictionCoefficient}}
({\it real}),
default value = {\tt 0.0 }
default unit = {\it hartrees/$\hbar$} ($41.3 PHz = 4.13 \times 10^{16} Hz$)

Viscosity parameter in molecular dynamics, in atomic units (inverse time).
If less than $10^{-6}$
{\it $hartrees/\hbar$} ( approximately 41 GHz ), then a micro-canonical ensemble is used: total energy is assumed
constant, and the atoms start moving with a velocity distribution consistent
with the initial temperature. If more, then a canonical ensemble is used and
true Langevin molecular dynamics is performed.

\subsubsection{\tt Step\_num 
\index{Step\_num@{\tt Step\_num}}
\label{Stepnum}}
({\it integer}),
default value = {\tt 250 }

Number of steps (iterations) in molecular dynamics. A successful
simulation ends only after reaching the requested number of steps.

\subsubsection{\tt Restart\_mode 
\index{Restart\_mode@{\tt Restart\_mode}}
\label{Restartmode}}
({\it boolean}),
default value = {\tt false }

Enables the restart of an aborted molecular dynamics simulation.
With this feature, the last two
sets of positions, velocities and accelerations stored in input file
{\tt mdinit.dat} are used. The number of steps in that file is included in the
current run as well. WARNING: make sure that the initial/final
temperatures and other MD parameters are correct in the restarted run.










\subsection{Polarizability}

\subsubsection{\tt Polarizability 
\index{Polarizability@{\tt Polarizability}}
\label{Polarizability}}
({\it boolean}),
default value = {\tt false }

Enables calculation of polarizability using a finite-field method. Available
only with non-periodic boundary conditions. An uniform electric field is
applied successively along the +x,-x,+y,-y,+z,-z directions, and dipole moments
are calculated and collected and printed in output file {\tt polar.dat}. From
dipole moments, the diagonal part of polarizability tensor is
obtained. Symmetry operations are ignored if this flag is used.

\subsubsection{\tt Electric\_Field 
\index{Electric\_Field@{\tt Electric\_Field}}
\label{ElectricField}}
({\it real}),
default value = {\tt 0.001},
{\it default unit = ry/bohr/e} (rydbergs/Bohr radii/electron charge)

Magnitude of external electric field to be applied.












\subsection{MPI parallelization options}


\subsubsection{\tt MPI\_Groups 
\index{MPI\_Groups@{\tt MPI\_Groups}}
\label{MPIGroups}}
({\it integer}),
default value = {\tt maximum valid value }

Number of groups (or subsets) of processors. PARSEC allows for
parallelization with respect to number of k-points $N_k$ (periodic systems
only), of spin channels $N_s$ (spin-polarized systems only) and of
symmetry representations $N_r$, besides the standard parallelization
with respect to grid points. Each triplet $(k,s,r)$ of k-point,
spin channel and representation is assigned to a single
group. Processors belonging to the same group work together in the
calculation of eigenvalues for the Hamiltonian $H_{k,s,r}$ assigned to
this group. The number of groups must be a factor of the total number
of processors ($N_p$) and of $N_k * N_s * N_r$, so that all groups have the
same number of processors. By default, the number of groups is chosen
to be the highest common factor between $N_p$ and $N_k * N_s *
N_r$. If a lower number is specified at input, the actual number of
groups may be reduced so that proper load balance is present.







\subsection{Flags for additional input/output}

\subsubsection{\tt Output\_All\_States 
\index{Output\_All\_States@{\tt Output\_All\_States}}
\label{OutputAllStates}}
({\it boolean}),
default value = {\tt false }

Forces output of all wave-functions up to requested number of states to file
{\tt parsec.dat}, instead of only the occupied ones (default).

\subsubsection{\tt Save\_Wfn\_Bands 
\index{Save\_Wfn\_Bands@{\tt Save\_Wfn\_Bands}}
\label{SaveWfnBands}}
({\it block}),
default value = {\tt no default } (optional)

By default, file {\tt parsec.dat} contains wave-functions of all occupied
energy levels. Alternatively, the user can select specific
wave-functions to be printed by giving a list of energy bands (levels,
for non-periodic systems) inside the block. For example, this:

\begin{verbatim}
begin save_wfn_bands
1
2
4
end save_wfn_bands
\end{verbatim}

will result in only bands 1,2, and 4 being printed out. The user can
also specify a range of bands by giving the initial and final
levels. This:

\begin{verbatim}
begin save_wfn_bands
2 8
end save_wfn_bands
\end{verbatim}

will result in all levels between 2 and 8 being printed. If no
wave-functions are to be printed, leave the block empty:

\begin{verbatim}
begin save_wfn_bands
end save_wfn_bands
\end{verbatim}

This option is ignored if the input {\tt Output\_All\_States  .true. }
is used. 

\subsubsection{\tt Save\_Intermediate\_Charge\_Density 
\index{Save\_Intermediate\_Charge\_Density@{\tt Save\_Intermediate\_Charge\_Density}}
\label{SaveIntermediateChargeDensity}}
({\it boolean}),
default value = {\tt false }

Causes wave-functions to be written to output after each self-consistent
iteration, instead of only after self-consistency is reached. Previous
wave-functions are over-written. This is useful for debugging purposes or
if diagonalization is too demanding.

\subsubsection{\tt Output\_Eigenvalues 
\index{Output\_Eigenvalues@{\tt Output\_Eigenvalues}}
\label{OutputEigenvalues}}
({\it boolean}),
default value = {\tt false }

Saves eigenvalues to file {\tt eigen.dat} after self-consistency in density
function is achieved. This is useful for debugging purposes.

\subsubsection{\tt Save\_Intermediate\_Eigenvalues 
\index{Save\_Intermediate\_Eigenstates@{\tt Save\_Intermediate\_Eigenvalues}}
\label{SaveIntermediateEigenvalues}}
({\it boolean}),
default value = {\tt false }

Saves eigenvalues after each SCF iteration, instead of after the last one.
Useful for debugging purposes.

\subsubsection{\tt Output\_Level 
\index{Output\_Level@{\tt Output\_Level}}
\label{OutputLevel}}
({\it integer}),
default value = {\tt 0 }

Output flag for additional info in parsec.out. With default value,
only the basic output is written. Positive values cause additional
information to be printed out:

\begin{verbatim}
> 0 : print out local/non-local components of force
> 1 : print out all symmetry data and additional eigensolver output
> 2 : print out more BFGS output
> 3 : print out all BFGS output
> 4 : print out all g-stars (warning: _lots_ of data)
> 5 : print out local potentials (warning: _lots_ of data)
\end{verbatim}

\subsubsection{\tt Output\_File\_Name 
\index{Output\_File\_Name@{\tt Output\_File\_Name}}
\label{OutputFileName}}
({\it word}),
default value = {\tt parsec.out }

Choice of output filename. Any name is valid, but spaces must not be
included, as well as some special symbols.

\subsubsection{\tt Restart\_From\_Wfndat 
\index{Restart\_From\_Wfndat@{\tt Restart\_From\_Wfndat}}
\label{RestartFromWfndat}}
({\it boolean}),
default value = {\tt false }

Makes possible to reuse an old file {\tt wfn.dat} in a PARSEC
run. Notice that file {\tt parsec.dat} is still being written, and
file {\tt wfn.dat} is not modified.









\section{Post-processing tools}

%\subsection{PVOX}
%
%PVOX (parsec visualization toolbox) is a set of matlab/python scripts,
%with graphical user interface. It was developed in 2005 by Michael
%Frasca, Lee Ballard, and Nick Voshell. It reads
%parsec output directly and extracts the density of states, makes plots of
%electron charge density, self-consistent potential, and selected
%wave-functions. It has been tested with \href{http://www.mathworks.com/products/matlab/}{Matlab 7.0} and Python 2.4.1.

\subsection{Other tools}

There are additional tools that read parsec output and perform specific
tasks. Some of them are: {\tt angproj}, for calculation of angle-projected
density of states; {\tt plotdx}, interface with
{Data Explorer}{http://www.opendx.org} for visualization
of charge density, wave-functions, etc.









\section{Problems, unfinished work, etc.}
\label{KnownProblems}









\section{Troubleshooting}

\begin{enumerate}
\item Diagla occasionally shows bad performance if periodic boundary
  conditions are used. If you get error messages related to this
  eigensolver, change it.

\item Missing symmetry operations: only symmetry operations belonging
  to the cubic system are tested for existence. That is because the
  finite grid spacing breaks all other symmetries (for example, the
  five-fold rotational symmetry in icosahedral molecules). In the
  Hamiltonian, only some of the present operations are used, and the
  selection is very stringent. If you feel not all possible symmetry
  operations are included in the Hamiltonian, try rotating or
  translating the whole system.

\item Unrecognized chemical element: Parsec has an internal periodic
  table (file {\tt ptable.f}), and it checks for the existence of
  input elements. If the
  calculation stops with "unknown element", edit the periodic table.

\item Current pseudopotential generator does not support asymptotically
   corrected functionals (LB, CS).

\item The EDSL library sometimes screws up when identifying
certain features with certain atom types. All the relevant information
about pseudopotentials is written to file {\tt parsec.out}. The user
should search that file for inconsistencies in input data.
\end{enumerate}








\section{Modifications}

\subsection{Modifications in version 1.2v1}

Version 1.2 is unstable and should be replaced with 1.2v1.
This is a list of modifications in version 1.2v1 compared to
version 1.1:

\begin{enumerate}
\item Two new eigensolvers are available: thick-restart Lanczos and
  Chebyshev-Davidson filtering. Eigensolvers implemented by Yunkai
  Zhou and tested by Yunkai and Murilo Tiago.

\item Chebyshev subspace filtering has been implemented. 
This feature should be always used since it improves considerably 
the solution of Kohn-Sham
equations at each SCF iteration. This was also implemented by Yunkai
  Zhou, and tested by Yunkai and Murilo Tiago.

\item Numerical tolerance in the search for symmetry operations has
  increased. Previous tolerance was causing crashes.

\item The output on file parsec.out has been modified in various places,
with minor changes.

\item Usage of "stop" statements was replaced with calls to exit\_err. This
avoids sudden crashes without any meaningful error message.
\end{enumerate}

\subsection{Modifications in version 1.2.7.2}

Compared to versions 1.2v1 and 1.2.7.1, this version has minor bug fixes and some
features added. The most important ones are:

\begin{enumerate}
\item Embedding of classical point charges ({\tt
  Add\_Point\_Charges} {AddPointCharges}.

\item Initial spin polarization for each chemical element ({\tt
  Initial\_Spin\_Polarization} {InitialSpinPolarization}.
\end{enumerate}

In addition, support for complex algebra is added to all eigensolvers
except TRLANC (version 1.2.7.1 also has it). This is a first step towards full support of k-point
sampling.

\subsection{Modifications in version 1.2.8}

Version 1.2.8 is an early merge of versions 1.2.4, 1.2.7.1 and
1.2.7.2. Compared to version 1.2.7.2, these are the most important
modifications:

\begin{enumerate}
\item Support for non-orthogonal unit cells (periodic systems) is
  implemented. Non-orthogonal unit vectors can be defined at input
  with the block {\tt Cell\_Shape} {CellShape} (notice that
  three lines must be given in the block, instead of one!). Some
  additional testing must be done for this feature.
\item Option for name of output file. See {\tt
 Output\_File\_Name} {OutputFileName}.
\item Wavefunctions are now printed out in file {\tt parsec.dat}
  instead of {\tt wfn.dat}. This modification is necessary because
  additional information must be saved, such as coordinates of unit
  lattice vectors. The user can still restart a calculation from
  a previous {\tt wfn.dat} file by using the input option {\tt
  Restart\_From\_Wfndat} {RestartFromWfndat}.
\item Choice of printing out some wave-functions to file {\tt
  parsec.dat}. See {\tt Save\_Wfn\_Bands} {SaveWfnBands}.
\end{enumerate}

Support for k-point sampling is not yet finished, but input variables
are already defined. See section ``Wave functions and k-points''.

Unresolved issue: the ARPACK eigensolver seems to occasionally miss eigenvalues
if complex algebra is turned on ({\tt Complex\_Wfn .true.}
{ComplexWfn}).

\subsection{Modifications in version 1.2.8.2}

Version 1.2.8.2 is a bug-fix branch that was created from
1.2.8. Modifications from version 1.2.8:

\begin{enumerate}
\item Double grid is implemented. See {\tt
 Double\_Grid\_Order} {DoubleGridOrder}.

\item Support for pseudopotential formats {\tt Martins\_Wang} and {\tt
  FHIPP} is added. See { input of pseudopotential files}
{InputPseudopotentials}.

\item Support for the ``LDA+U'' approximation is added. See
  {\tt LDAplusU\_U LDAplusU\_J }{LDAplusUU}.

\item Support for spin-orbit calculations. This is still in development stage.

\end{enumerate}

\subsection{Modifications in version 1.2.9}

Version 1.2.9 is the first version of this code with support for k-points
different from the $\Gamma$ point. More details about the input of
k-points is found in section
{``Wave functions and k-points'' }{KpointMethod}. Support for
spin-orbit calculations, multiple k-points and LDA+U is still under
tests.

\subsection{Modifications in version 1.2.9.2}

Version 1.2.9.2 has more stable support for spin-orbit calculations,
multiple k-points and LDA+U. In addition, it has two new input
options: {\tt Kpoint\_Unit}{KpointUnit} and {\tt
  Boundary\_Mask}{BoundaryMask}.

\subsection{Modifications in version 1.2.9.3}

Version 1.2.9.3 has an alternative choice for the parallelization:
besides the standard parallelization with respect to grid points, this
version can also work with parallelization with respect to number of
k-points, of spin channels and of symmetry representations. See {\tt
  MPI\_Groups}{MPIGroups}. In addition, the option of interrupting
``smoothly'' a calculation is added. See section { ``Interrupting
  calculation''} {StopSCF}.

\subsection{Modifications in version 1.2.9.4}

Wire boundary conditions are implemented. See {\tt
  Boundary\_Conditions}{BoundaryConditions}. Also, all source files
  were converted to free form, except for the LBFGS library.

\subsection{Modifications in version 1.3}

\begin{itemize}
\item Inclusion of multi-secant mixing methods, as developed by Haw-ren Fang
and Yousef Saad, from EECS, Univ. of Minnesota.

\item Addition of restart from previous spin-orbit calculation. Fixed a bug
in the write-out of spin-orbit wavefunctions. Older versions do not write
correct spin-orbit wavefunctions.

\item Removal of old/unused input flags: skip\_rr\_rotation, skip\_force,
spinorbit, old\_pseudopotential\_format.

\item Addition of a gfortran makefile. Removal of unused variables and
commented-out old coding.

\item Inclusion of time-reversal symmetry in Monkhorst-Pack k-grid.
\end{itemize}

\subsection{Modifications in version 1.3.4}

\begin{itemize}
\item A number of bugs were fixed:

\begin{enumerate}
\item Definition of non-orthogonal directions in periodic systems has
  been modified. This is an attempt to fix the problem with incorrect
  ground state in BCC crystals.

\item Bugs related to the definition of grid spacing in periodic
  systems, reading of old charge density (it used to crash in wires),
  calculation of total energy in wires, and array-out-of-bound errors
  in BFGS relaxation were all fixed.
\end{enumerate}

\item A cubic spline interpolation was implemented. See
  \ref{CubicSpline}.

\item Functional Perdew-Zunger spin-polarized (LSDA) was added. For
  the sake of back-compatibility, the default LSDA functional is still
  Perdew-Wang.
\end{itemize}

\addcontentsline{toc}{section}{Index}

\end{document}
